% ============================================================================
% !Document: IM01 [zxweather Installation Reference]
% ----------------------------------------------------------------------------
% !Revision: 001
% !IssueDate: July 2012
% !Status: Unreleased
%
% !-Classification
% !ProjectCode: DAZW [Database Applications, zxweather]
% !Type: IG [Installation Guide]
%
% !Copyright: (C) David Goodwin, 2012
% !License: FDL [GNU Free Documentation License]
% !Auhtor: David Goodwin
% ============================================================================

% Document information. This should match the above
\newcommand{\doctitle}{zxweather}
\newcommand{\docsubtitle}{Installation Reference}
%\newcommand{\projectnum}{DAZW}
\newcommand{\docnum}{IG01}
\newcommand{\docrev}{001}
\newcommand{\docdate}{July 2012}
\newcommand{\docauthor}{David Goodwin}
\newcommand{\docabstract}{This document provides installation and configuration guidelines for zxweather.}
\newcommand{\docupdateinfo}{This is a DRAFT DOCUMENT.}
\newcommand{\docosver}{Microsoft Windows NT 5.0+; Linux.}
\newcommand{\docswver}{zxweather 1.0}
\newcommand{\doccopyright}{\textcircled{c} Copyright David Goodwin, 2012.}
\newcommand{\doclicense}{Use, reproduction and modification of this document is permitted subject to the terms of the GNU Free Documentation License, Version 1.3 or any later vesion published by the Free Software Foundation. See \url{http://www.gnu.org/copyleft/fdl.html} for full license text.}

%%%%%%%%%%%%%%%%%%%%%%%%%%%%%%%%%%%%%%%%%%%%%%%%%%%%%%%%%%%%%%%%%%%%%%%%%%%%%
%                                 CONFIGURATION                             %
%%%%%%%%%%%%%%%%%%%%%%%%%%%%%%%%%%%%%%%%%%%%%%%%%%%%%%%%%%%%%%%%%%%%%%%%%%%%%

% Book type document, A4 paper, 10pt std font size:
\documentclass[a4paper,10pt,draft]{book} 

\usepackage[scaled=0.90]{helvet} % Use helvetica as the standard font
\usepackage{courier}			 % Use courier as the fixed-width font
\usepackage{hyperref}			 % Links in PDF output
\usepackage{a4wide}				 % Narrower margins for A4 documents
\usepackage{ifthen}				 % A few if statements
\usepackage{multirow}			 % Column spanning in tables


% use zxtechdoc styles if they're there
\IfFileExists{zxtitle.sty}{\usepackage{zxtitle}}{}
\IfFileExists{zxtechdoc.sty}{\usepackage{zxtechdoc}}{}

\hypersetup{
    unicode=false,          % non-Latin characters in Acrobat’s bookmarks
    pdftoolbar=true,        % show Acrobat’s toolbar?
    pdfmenubar=true,        % show Acrobat’s menu?
    pdffitwindow=false,     % window fit to page when opened
    pdfstartview={FitH},    % fits the width of the page to the window
    pdftitle={\doctitle{} - \docsubtitle},    % title
    pdfauthor={\docauthor},     % author
    pdfsubject={\doctitle},   % subject of the document
    pdfkeywords={\doctitle} {\docsubtitle} {key3}, % list of keywords
    pdfnewwindow=true,      % links in new window
    colorlinks=true,       % false: boxed links; true: colored links
    linkcolor=black,          % color of internal links
    citecolor=green,        % color of links to bibliography
    filecolor=magenta,      % color of file links
    urlcolor=cyan           % color of external links
}

% Build the partnumber. Format is PROJ-DOCU.REV. If revision is 001 then it 
% is not displayed. If project is undefined it is not displayed.
\newcommand{\partnumber}{\ifthenelse{\isundefined{\projectnum}}{}{\projectnum-\docnum	\ifthenelse{\equal{\docrev}{001}}{}{.\docrev}}}

\begin{document}

% Roman Numerals for the front matter
\pagenumbering{roman}

% Setup the titlepage. We will use the zxtitle format if its there,
% otherwise the much simpler standard LaTeX one.
\ifthenelse{\isundefined{\ordernumber}}{

% Standard LaTeX titlepage
\title{\doctitle{} - \docsubtitle}
\author{\docauthor}
}{

% zxtitle titlepage
\title{\doctitle}
\subtitle{\docsubtitle}
\titleabstract{\docabstract}
\ordernumber{\partnumber}
\updateinfo{\docupdateinfo}
\osinfo{\docosver}
\swversion{\docswver}
\titlecopyright{\doccopyright}
\licensestatement{\doclicense}
}
\date{\docdate}

\maketitle

\clearpage

\tableofcontents
\clearpage

%%%%%%%%%%%%%%%%%%%%%%%%%%%%%%%%%%%%%%%%%%%%%%%%%%%%%%%%%%%%%%%%%%%%%%%%%%%%%
%                                 DOCUMENT                                  %
%%%%%%%%%%%%%%%%%%%%%%%%%%%%%%%%%%%%%%%%%%%%%%%%%%%%%%%%%%%%%%%%%%%%%%%%%%%%%

%TODO : Normalise terminology across documents (use either 'sample' or 'history record').

\chapter{Introduction}
% Back to arabac numerals for the proper content
\pagenumbering{arabic}
\setcounter{page}{1}

zxweather is software for automatic weather stations. It is licensed under the GNU GPL making it free software. Its main features are:
\begin{itemize}
\item A modern HTML5 web interface
\item Basic HTML fallback for older browsers.
\item Full weather history
\end{itemize}

This manual provides instructions to install and configure all components of the system. These are:
\begin{itemize}
\item Database
\item Data logger
\item Web interface
\end{itemize}

It is recommended that you read this manual in its entirety. It is the only documentation for installing and configuring all of zxweather.

\section{Supported Hardware}
At this time only weather stations 100\% compatible with the Fine Offset WH1080 are supported.

\section{System Requirements}
All components of zxweather are supported under modern versions of both Linux and Windows. It is recommended that Linux be used if possible however.

PostgreSQL v9.0+ is the only supported database engine. Other RDBMS such as MySQL are not supported in any way. Older versions of PostgreSQL may work but have not been tested.

The data collection tools are only supported on little-endian CPU architectures at this time. This includes Intel IA32 (x86) and most ARM processors. Bad things will happen if your processor is big-endian.

\subsection{Data logger}
The data collection tools are written in the C language and are portable to both Linux and Windows.

ECPG (a PostgreSQL utility), GCC and GNU Make are required to compile the software. To run it, libecpg and (on linux) libusb-1.0 are required.

\subsection{Web interface}
The web interface requires Python version 2.7 with the following packages:
\begin{itemize}
\item Jinja2
\item web.py
\item psycopg2
\end{itemize}

Additionally, gnuplot must be available on the system running the weatherplot program.

\section{Related Documents}
\begin{tabular}{l l}
\verb|DAZW-WG01| & WH1080 Utilities Users Guide, version 1.0 \\
\verb|DAZW-DB01| & zxweather Database Structure, version 1.0 \\
\end{tabular}

\chapter{Database}
\label{cha_database}
% Instructions for how to setup the database. This includes permissions
% required, etc.

This chapter describes the database setup required by zxweather. The database is required by the core system and is not optional. It must be setup on a system that all other components have network access to.

\section{Installation}

%TODO Come up with a better database setup system. Write a python program or something.

The zxweather database has only been tested with PostgreSQL version 9.0 and above. This documentation assumes you already have a suitable version of the server and client tools installed on your system.

This section documents creating the zxweather using the command-line PostgreSQL tools. If you are more comfortable with the pgAdmin III GUI tool you may use that instead.

\subsection{Create Database}
To create the weather database, execute a command such as the following:

\begin{verbatim}
$ createdb -h dbserver -U username weather "Weather database"
\end{verbatim}

Where:
\begin{itemize}
\item \emph{dbserver} is the hostname (or IP Address) of the machine running PostgreSQL.
\item \emph{username} is the username to login to the server with. This will often be something like "postgres". The account being used must have the \emph{CREATEDB} permission.
\item \emph{weather} is the name of the new database.
\item \emph{"Weather Database"} is a description of the database. This is optional.
\end{itemize}

More information on the createdb program can be found at \url{http://www.postgresql.org/docs/9.1/static/app-createdb.html}.

\subsection{Create Database Structure}
The database structure needed by zxweather to store data is created using the \verb|database.sql| located in the \emph{database} directory of the zxweather distribution. Running this SQL script will create the full database structure in one step.

You can run this script with a command such as the following:
\begin{verbatim}
$ psql -h dbserver -U username -d weather -f database.sql 
\end{verbatim}
This command must be run from inside the database directory. The supplied parameters are:
\begin{itemize}
\item \emph{dbserver} - The hostname (or IP Address) of the machine running PostgreSQL
\item \emph{username} - The user account to login to the server with. Using a superuser account (often called "postgres") will be easiest.
\item \emph{weather} - The name of the database you created in the previous section.
\end{itemize}

\section{Permissions}
Four programs will be accessing the database. They require user accounts with the following permissions on the weather database:

\begin{tabular}{l l}
\hline
\textbf{Program} & \textbf{Permissions} \\
\hline
wh1080d & CONNECT, SELECT, INSERT, UPDATE \\
wh1080 & CONNECT, SELECT, INSERT \\
weatherplot & CONNECT, SELECT \\
web interface & CONNECT, SELECT \\
\hline
\end{tabular}

You can create individual accounts for each program or have them all sharing the same account.

%TODO Instructions for creating the accounts

\chapter{Data Logger}
% Compiling and installing wh1080d or the other program

Before starting the wh1080d program you must perform a full data load if any of the following conditions are true and your database is not empty. 
\begin{itemize}
\item \verb|wh1080| - The WH1080 Update Tool
\item \verb|wh1080d| - The WH1080 Update Service
\end{itemize}

The \emph{Update Tool} allows you to inspect configuration and data on the device (including exporting it as CSV or inserting it into the database). It is useful for performing some maintenance tasks and troubleshooting.

The \emph{Update Service} runs continuously feeding both samples and live data into the database for use by other programs. This program runs as service under Windows or a daemon under Linux.

These tools are covered in more detail by the \emph{WH1080 Utilities Users Guide, version 1.0} (\verb|DAZW-WG01|).

Use of these tools requires that the database has already been setup. If you have not done this yet consult chapter \ref{cha_database}.

\section{Compiling}
This section only covers compiling the tools under Linux. If you are using Microsoft Windows it is recommended that you use the pre-compiled executables and skip to the next section.

\subsection{Requirements}
To compile the tools under Linux the following software must be installed:
\begin{itemize}
\item GNU Make
\item GNU C Compiler
\item ECPG (a PostgreSQL utility)
\end{itemize}

The following libraries are also required:
\begin{itemize}
\item libpq (PostgreSQL client library)
\item libecpg (library for ECPG)
\item libusb-1.0
\end{itemize}

These libraries and software packages should be available from your operating systems package manager. On Debian 6.0 the packages would be \verb|build-essential libusb-1.0-0-dev libecpg-dev|.

\subsection{Compiling the WH1080 Tools}

%TODO Switch to the GNU Build System and rewrite this.

To compile the WH1080 tools, cd into the wh1080 subdirectory of the zxweather distribution and run \verb|make|. This should kick off the build process and leave you with a handful of programs.


\section{Loading Data}

Before you can start the update service it is important to note that you may have to run a \emph{Full Update}. It is important to do this to do this when necessary to prevent:
\begin{itemize}
\item Corrupt data being inserted into the database
\item New data on the weather station from being lost
\end{itemize}

However, if the a Full Update is performed when it is \emph{not} necessary it will result in duplicate samples being inserted into your database.

\subsection{When to perform a Full Update}
There are only three occasions when it is acceptable (and necessary) to perform a full update:
\begin{itemize}
\item You have just erased the weather stations memory
\item You have reset the weather station
\item The database is more than 4080 samples out of date
\end{itemize}

The first two deal with the case where the sample the database says needs to be downloaded next no longer exists on the weather station. This prevents invalid data from being inserted into the database.

The third deals with the case where the sample the database says needs to be downloaded next is not correct because \emph{all} samples needed to be downloaded. Failing to perform a full update in this case will likely result in some of the samples being missed

As an example, if your weather station is set to take one sample every five minutes then the database must be more than 340 hours out of date (that is a little over 14 days).


\subsection{Performing a Full Update}
To perform a full update you must use the Update Tool (the wh1080 program) as the Update Service can not perform this operation.

The \verb|-l| command-line option causes the Update Tool to perform a Full Update instead of a regular one:

\verb|wh1080 -l -d databasename@hostname -u username -p password|

This is covered in more detail in Section 2.2.4 of \emph{WH1080 Utilities Users Guide} (DAZW-WG01).

\section{Update Service Configuration}
\subsection{Linux}

The update service takes the following arguments:

\begin{tabular}{l l p{10cm}}
\hline
\textbf{Argument} & \textbf{Parameter} & \textbf{Description} \\
\hline
-d & database & Database connection string \\
-u & username & Database username \\
-p & password & Database password \\
-f & filename & Log file to write messages to \\
\hline
\end{tabular}

Under linux the Update Service runs as a daemon. To start it just run something like:

\verb|wh1080d -d database -u username -p password -f logfile|

The log file is truncated when the daemon starts.

\subsection{Windows}


\chapter{Web Interface}

The web interface is the primary way for viewing data in the zxweather database. It consists of two components - the web application and the chart plotting program.

Both components are required for correct operation. This chapter describes how to install and configure them.

% Overview of features


\section{WSGI Application}

The web interface is written as a Python WSGI application. This section only covers installing the application under Apache httpd. If you are not using Apache then consult your web servers documentation for instructions on installing wsgi applications.

\subsection{Installation}
% Where to put it
% Apache configuration

At this time zxweather cannot be run in a subdirectory - it must live in the root directory of the website. This generally means giving it its own virtual host.

The system you are installing the web interface on must have the following installed on it:
\begin{itemize}
\item Apache httpd
\item mod\_wsgi
\item Python 2.7
\item psycopg2
\item web.py
\end{itemize}

Installation and configuration of these packages is outside the scope of this document. If you are setting up on a Linux system the packages are likely all available from your distributions package repositories.

Where examples are provided in this section they are for Debian-based Linux distributions. Installation and configuration on Windows systems is similar but requires more effort.

\subsubsection{Extract Distribution}
Firstly extract the full zxweather documentation to some location on your disk. In the examples \\ /var/zxweather is used:
\begin{verbatim}
$ pwd
/var/zxweather
$ ls
database  doc  index.txt  plot  wh1080  zxw_web
\end{verbatim}

\subsubsection{WSGI Setup}
As the zxweather web interface is a Python WSGI application you must have the mod\_wsgi installed and enabled. On Debain-based systems the package is called \verb|libapache2-mod-wsgi| and can be enabled using the \verb|a2enmod| tool:

\begin{verbatim}
$ a2enmod wsgi
\end{verbatim}

\subsubsection{Virtual-host Configuration}

Then add something like the following to your Apache configuration:
\begin{verbatim}
WSGIScriptAlias / /var/zxweather/zxw_web/zxweather.py
\end{verbatim}


\subsection{Configuration}
% The configuration file
% About page

\section{Chart Plotting}
% Purpose of this program
% Command-line arguments
% How to set it up

The \verb|weatherplot| program is responsible for generating static charts primarily used by the basic HTML web interface. It must be setup as a scheduled task to be run at regular intervals.

The machine it runs on must have \emph{gnuplot} installed and must have access to the database server and directory the zxweather web interface runs from.

\subsection{Usage}

The weatherplot program is written in the Python language and lives in the \verb|plot| subdirectory of the zxweather distribution. It is executed on the command-line with python as: \\ \verb|$ python plot/weatherplot.py [arguments]|.

The charts it generates must be put in the web interfaces \emph{static data} directory in a subdirectory with the same name as your station. For example, if your station is named "foo" and the zxweather distribution was extracted to \verb|/var/zxweather| then you would generate charts into \\ \verb|/var/zxweather/zxw_web/static/foo/|.

By default an executable called \verb|gnuplot| is expected to be in the path which it can use to generate the charts. If your install of gnuplot is not in the path or goes by another name use the \verb|--gnuplot-binary| parameter to specify its name.

\subsubsection{Example}
This example regenerates \emph{all} charts for all data in the database. It is run from the zxweather directory.
\begin{verbatim}
$ python plot/weatherplot.py --directory zxw_web/static/foo/ \
-t weather -n localhost -u postgres -p password
\end{verbatim}

If your install of gnuplot is not in the path or is not called \verb|gnuplot| then you would supply the \\ \verb|--gnuplot-binary| argument:
\begin{verbatim}
$ python plot/weatherplot.py --directory zxw_web/static/foo/ \
-t weather -n localhost -u postgres -p password \
--gnuplot_binary /opt/gnuplot/bin/gnuplot-custom-build
\end{verbatim}

\subsubsection{Command-line Arguments}
The weatherplot program accepts the following command-line arguments:

\begin{tabular}{p{3.4cm} l p{8cm}}
\hline
\textbf{Argument} & \textbf{Parameter} & \textbf{Description} \\
\hline
\verb|-t| \par \verb|--database| & dbname & Name of the database to use. Required. \\
\verb|-n| \par \verb|--host| & hostname & Database server hostname. Required. \\
\verb|-u| \par \verb|--user| & username & Username for database server. Required. \\
\verb|-p| \par \verb|--password| & password & Password for database server. Required. \\
\verb|-d| \par \verb|--directory| & directory & Output directory. Required. \\
\verb|-a| \par \verb|--plot-new| & filename & Only plots charts for dates on or after that stored in the specified file. \\
\verb|-r| \par \verb|--replot-pause| & seconds & Number of seconds to wait before replotting.\\
\verb|-g| \par \verb|--gnuplot-binary| & filename & Name of the gnuplot executable to use.\\
\hline
\end{tabular}

The database, host, user, password and directory parameters are always required.

\subsection{Plotting All Charts}

To regenerate charts for your entire database run the weatherplot program with only the minimum command-line arguments:

\begin{verbatim}
$ python plot/weatherplot.py --database weather --host localhost \
--user postgres --password password \
--directory zxw_web/static/station_name/
\end{verbatim}

This will create charts for all days and months in your database and store them in the specified directory. Depending on the size of your database this may take some time.

Some software upgrades may require you to do this when new chart types have been added or the style of the charts has been adjusted.

\subsection{Running as a Scheduled Task}

The recommended way to setup the weatherplot program is to run it as a scheduled task from cron or the windows task scheduler. When run in this way it is important that it be set to only plot charts containing new data.

The \verb|--plot-new| command-line argument implements this. The argument takes a single parameter which is the name of a file to store the date of the last plotted day in.

Each time the weatherplot program is executed with this parameter it will replot all charts for all days and months on or after the date in that file and then update the file with todays date. That way only charts that need to be regenerated are regenerated.

\subsubsection{Example}

When run as below weatherplot will only replot charts that have changed since it was last run:
\begin{verbatim}
$ python plot/weatherplot.py --database weather --host localhost
--user postgres --password password --directory static/station_name/
--plot-new plot_status_file
\end{verbatim}

To make this run every 30 minutes you would add a line such as the following to /etc/crontab:
\begin{verbatim}
0,30 *  * * *   root    cd /var/zxweather && \
  python plot/weatherplot.py -t weather -n localhost -u postgres \
  -p password -d zxw_web/static/station_name -a plot_status_file  
\end{verbatim}

\subsection{Plotting Continuously}

The weatherplot program is capable of running interactively in continuous mode. When run like this it will automatically replot charts at a specific interval until you terminate it with Ctrl+C. This is primarily intended for testing purposes.

It can be run in this mode by supplying the \verb|--replot-pause| parameter
with a suitable interval in seconds.

\subsubsection{Example}

When run as below the behaviour is the same as setting it up to be run by cron every 30 minutes except it runs continuously attached to the terminal.

\begin{verbatim}
$ python plot/weatherplot.py --database weather --host localhost \
--user postgres --password password --directory zxw_web/static/rua \
--plot-new plot_status_file --replot-pause 1800

Weather data plotting application v1.0
        (C) Copyright David Goodwin, 2012


Connecting to database...
Server version: PostgreSQL 9.1.2, compiled by Visual C++ build 1500
Generating temperature plots in zxw_web/static/rua
Plotting from 2012-05-10
Plotting graphs for 2012...
Plotting graphs for 2012 may...
Plot zxw_web/static/rua/2012/may/temperature_tdp_large.png
Plot zxw_web/static/rua/2012/may/temperature_awc_large.png
Plot zxw_web/static/rua/2012/may/humidity_large.png
Plot zxw_web/static/rua/2012/may/indoor_humidity_large.png
Plot zxw_web/static/rua/2012/may/pressure_large.png
Plot zxw_web/static/rua/2012/may/indoor_temperature_large.png
Plot zxw_web/static/rua/2012/may/temperature_tdp.png
Plot zxw_web/static/rua/2012/may/temperature_awc.png
Plot zxw_web/static/rua/2012/may/humidity.png
Plot zxw_web/static/rua/2012/may/indoor_humidity.png
Plot zxw_web/static/rua/2012/may/pressure.png
Plot zxw_web/static/rua/2012/may/indoor_temperature.png
Plotting graphs for 2012 may 9...Skip
Plotting graphs for 2012 may 10...
Plot zxw_web/static/rua/2012/may/10/temperature_tdp_large.png
Plot zxw_web/static/rua/2012/may/10/temperature_awc_large.png
Plot zxw_web/static/rua/2012/may/10/humidity_large.png
Plot zxw_web/static/rua/2012/may/10/indoor_humidity_large.png
Plot zxw_web/static/rua/2012/may/10/pressure_large.png
Plot zxw_web/static/rua/2012/may/10/indoor_temperature_large.png
Plot zxw_web/static/rua/2012/may/10/temperature_tdp.png
Plot zxw_web/static/rua/2012/may/10/temperature_awc.png
Plot zxw_web/static/rua/2012/may/10/humidity.png
Plot zxw_web/static/rua/2012/may/10/indoor_humidity.png
Plot zxw_web/static/rua/2012/may/10/pressure.png
Plot zxw_web/static/rua/2012/may/10/indoor_temperature.png
Plot zxw_web/static/rua/2012/may/10/7-day_temperature_tdp_large.png
Plot zxw_web/static/rua/2012/may/10/7-day_temperature_awc_large.png
Plot zxw_web/static/rua/2012/may/10/7-day_humidity_large.png
Plot zxw_web/static/rua/2012/may/10/7-day_indoor_humidity_large.png
Plot zxw_web/static/rua/2012/may/10/7-day_pressure_large.png
Plot zxw_web/static/rua/2012/may/10/7-day_indoor_temperature_large.png
Plot zxw_web/static/rua/2012/may/10/7-day_temperature_tdp.png
Plot zxw_web/static/rua/2012/may/10/7-day_temperature_awc.png
Plot zxw_web/static/rua/2012/may/10/7-day_humidity.png
Plot zxw_web/static/rua/2012/may/10/7-day_indoor_humidity.png
Plot zxw_web/static/rua/2012/may/10/7-day_pressure.png
Plot zxw_web/static/rua/2012/may/10/7-day_indoor_temperature.png
Plot completed at 2012-05-10 22:31:46.953000
Waiting for 1800 seconds to plot again. Press Ctrl+C to terminate.
\end{verbatim}

%%%%%%%%%%%%%%%%%%%%%%%%%%%%%%%%%%%%%%%%%%%%%%%%%%%%%%%%%%%%%%%%%%%%%%%%%%%%%
%                                END DOCUMENT                               %
%%%%%%%%%%%%%%%%%%%%%%%%%%%%%%%%%%%%%%%%%%%%%%%%%%%%%%%%%%%%%%%%%%%%%%%%%%%%%

% Back page
\newpage
\thispagestyle{empty}
\begin{flushright}
\null
\vfill
\tt \partnumber
\end{flushright}
\end{document}